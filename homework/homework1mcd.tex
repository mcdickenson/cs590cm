\documentclass[12pt,letterpaper]{article}
\usepackage{amsmath,amsthm,amsfonts,amssymb,amscd}
\usepackage{fullpage}
\usepackage{graphicx}
\usepackage{lastpage}
\usepackage{enumerate}
\usepackage{fancyhdr}
\usepackage{hyperref}
\usepackage{mathrsfs}
\usepackage{xcolor}
\usepackage[margin=3cm]{geometry}
\setlength{\parindent}{0.0in}
\setlength{\parskip}{0.05in}

% Edit these as appropriate
\newcommand\course{CS590.04}
\newcommand\semester{Spring 2014}     % <-- current semester
\newcommand\hwnum{1}                  % <-- homework number
\newcommand\yourname{Matt Dickenson} % <-- your name
\newcommand\login{mcd31}           % <-- your NetID
\newcommand\hwdate{Due: 7 February, 2014}           % <-- HW due date

\newenvironment{answer}[1]{
  \subsubsection*{Problem #1}
}


\pagestyle{fancyplain}
\headheight 35pt
\lhead{\yourname\ \texttt{\login}\\\course\ --- \semester}
\chead{\textbf{\Large Homework \hwnum}}
\rhead{\hwdate}
\headsep 10pt

\begin{document}

% \noindent \emph{Homework Notes:} 

\begin{answer}{1}
\paragraph{a} 
\framebox{\parbox{\textwidth}{A gamble that offered \$2,000 with probability $\frac{1}{5}$ and \$-500 with probability $\frac{4}{5}$ would strictly increase Bob's expected utility.}} With probability $\frac{1}{5}$ he ends up with \$3,500 and has a utility of 2, and with probability $\frac{4}{5}$ he ends up with \$1,000, which has a utility of 1. Thus, Bob's expected utility is $\frac{1}{5} \cdot 2 + \frac{4}{5} \cdot 1 = \frac{6}{5} > 1$. This gamble is fair because the expected value is $\frac {1}{5} \cdot \$2,000 + \frac{4}{5} \cdot \$500 = \$400 + \$-400 = 0$. 

\paragraph{b}
\framebox{\parbox{\textwidth}{A gamble that offered \$500 with probability $\frac{3}{4}$ and \$-1,500 with probability $\frac{1}{4}$ would strictly decrease Bob's expected utility.}} With probability $\frac{3}{4}$ he would end up with \$2,000 ($u(b)=1$) and with probability $\frac{1}{4}$ Bob ends up with \$0 ($u(b)=0$). Thus, his expected utility is $\frac{3}{4} \cdot 1 + \frac{1}{4} \cdot 0 = \frac{3}{4} < 1$. This is a fair gamble because the expected value is $\frac{3}{4} \cdot \$500 + \frac{1}{4} \$-1,500 = \$375 + \$-375 = 0$. 
\end{answer}


\begin{answer}{2}
\end{answer}


\begin{answer}{3}

\begin{table}[h!]
\begin{center}
\begin{tabular}{l|c|c|c|}
\multicolumn{1}{c}{} & \multicolumn{1}{c}{Left} & \multicolumn{1}{c}{Center} & \multicolumn{1}{c}{Right} \\
\cline{2-4}
Top    & 5, 0 & 1, 2 & 4, 0 \\
\cline{2-4}
Middle & 2, 4 & 2, 4 & 3, 5 \\
\cline{2-4}
Bottom & 0, 1 & 4, 0 & 4, 0 \\
\cline{2-4}
\end{tabular}
\caption{Normal-form game for Problem 3(a) with labeled actions.}
\label{tab:3a}
\end{center}
\end{table}

\paragraph{a}
Table \ref{tab:3a} represents the normal-form game for this problem with the additional convenience of labeled strategies. Neither player has a strictly dominant pure strategy. Additionally, the column player does not have a strictly dominant mixed strategy: no mix over L and R can dominate C (if the row player plays T), no mix over L and C can dominate R (if the row player plays M), and no mix over C and R can dominate L (if the row player plays B). 

For the row player, no mixed strategy over M and B can dominate T (if the column player plays L) and no mixed strategy over T and M can dominate B (if the column player plays C), but M is dominated by a mix over T and B (e.g. $p_T=\frac{2}{5}, p_B=\frac{3}{5}$).

We can now solve the subgame with row M eliminated, and see that a mix over L and C will dominate R for the column player. There are no dominant pure strategies in this subgame, so we must find the mixed strategy equilibrium. The column player mixes such that
\begin{eqnarray*}
5 p_L + 1 p_C &=& 0 p_L + 4 p_C \\
5 p_L &=& 3 p_C \\
p_L = \frac{3}{8} &,& p_C = \frac{5}{8},
\end{eqnarray*}

and the row player mixes such that

\begin{eqnarray*}
0 p_T + 1 p_B &=& 2 p_T + 0 p_B \\
p_B = \frac{2}{3} &,& p_T = \frac{1}{3},
\end{eqnarray*}
both of which make the other player indifferent. 

\framebox{\parbox{\textwidth}{The unique Nash equilibrium is the row player mixing between T and B with probability ($\frac{1}{3}, \frac{2}{3}$), and the column player mixing between L and C with probability ($\frac{3}{3}, \frac{5}{8}$).}}

\paragraph{b}
\end{answer}


\begin{answer}{4}
\paragraph{a} The normal-form representation of this game is presented in Table \ref{tab:4a}.

\begin{table}[h!]
\begin{center}
\begin{tabular}{l|c|c|}
\multicolumn{1}{c}{} & \multicolumn{1}{c}{L} & \multicolumn{1}{c}{R}  \\
\cline{2-3}
L & 5, 2 & 0, 3 \\
\cline{2-3}
M & 0, 3 & 5, 2 \\
\cline{2-3}
R & 3, 2 & 1, 1 \\
\cline{2-3}
\end{tabular}
\caption{Normal-form game for Problem 4(a) with Player 1 on the row and Player 2 on the column.}
\label{tab:4a}
\end{center}
\end{table}

\end{answer}

\end{document}
