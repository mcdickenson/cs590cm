\documentclass[12pt,letterpaper]{article}
\usepackage{amsmath,amsthm,amsfonts,amssymb,amscd}
\usepackage{fullpage}
\usepackage{graphicx}
\usepackage{lastpage}
\usepackage{enumerate}
\usepackage{fancyhdr}
\usepackage{hyperref}
\usepackage{listings}
\usepackage{mathrsfs}
\usepackage{xcolor}
\usepackage[margin=3cm]{geometry}
\setlength{\parindent}{0.0in}
\setlength{\parskip}{0.05in}


% Edit these as appropriate
\newcommand\course{CS590.04}
\newcommand\semester{Spring 2014}     % <-- current semester
\newcommand\hwnum{1}                  % <-- homework number
\newcommand\yourname{Matt Dickenson} % <-- your name
\newcommand\login{mcd31}           % <-- your NetID
\newcommand\hwdate{Due: 12 February, 2014}           % <-- HW due date

\newenvironment{answer}[1]{
  \subsubsection*{Problem #1}
}


\pagestyle{fancyplain}
\headheight 35pt
\lhead{\yourname\ \texttt{\login}\\\course\ --- \semester}
\chead{\textbf{\Large Homework \hwnum}}
\rhead{\hwdate}
\headsep 10pt

\begin{document}

% \noindent \emph{Homework Notes:} 

\begin{answer}{1}
\paragraph{a} 
\framebox{\parbox{\textwidth}{A gamble that offered \$2,000 with probability $\frac{1}{5}$ and \$-500 with probability $\frac{4}{5}$ would strictly increase Bob's expected utility.}} With probability $\frac{1}{5}$ he ends up with \$3,500 and has a utility of 2, and with probability $\frac{4}{5}$ he ends up with \$1,000, which has a utility of 1. Thus, Bob's expected utility is $\frac{1}{5} \cdot 2 + \frac{4}{5} \cdot 1 = \frac{6}{5} > 1$. This gamble is fair because the expected value is $\frac {1}{5} \cdot \$2,000 + \frac{4}{5} \cdot \$500 = \$400 + \$-400 = 0$. 

\paragraph{b}
\framebox{\parbox{\textwidth}{A gamble that offered \$500 with probability $\frac{3}{4}$ and \$-1,500 with probability $\frac{1}{4}$ would strictly decrease Bob's expected utility.}} With probability $\frac{3}{4}$ he would end up with \$2,000 ($u(b)=1$) and with probability $\frac{1}{4}$ Bob ends up with \$0 ($u(b)=0$). Thus, his expected utility is $\frac{3}{4} \cdot 1 + \frac{1}{4} \cdot 0 = \frac{3}{4} < 1$. This is a fair gamble because the expected value is $\frac{3}{4} \cdot \$500 + \frac{1}{4} \$-1,500 = \$375 + \$-375 = 0$. 
\end{answer}

The margin for the first and third preference listings are satisfied by more than necessary.

\begin{answer}{2}

\lstinputlisting[caption=MathProg Code]{../code/hw1-util-estimator.mod}

\lstinputlisting[caption=MathProg Output]{../code/hw1-util-estimator.out}
\end{answer}


\begin{answer}{3}

\paragraph{a}
Table \ref{tab:3a} represents the normal-form game for this problem with the additional convenience of labeled strategies. Neither player has a strictly dominant pure strategy. Additionally, the column player does not have a strictly dominant mixed strategy: no mix over L and R can dominate C (if the row player plays T), no mix over L and C can dominate R (if the row player plays M), and no mix over C and R can dominate L (if the row player plays B). 

\begin{table}[h!]
\begin{center}
\begin{tabular}{l|c|c|c|}
\multicolumn{1}{c}{} & \multicolumn{1}{c}{Left} & \multicolumn{1}{c}{Center} & \multicolumn{1}{c}{Right} \\
\cline{2-4}
Top    & 5, 0 & 1, 2 & 4, 0 \\
\cline{2-4}
Middle & 2, 4 & 2, 4 & 3, 5 \\
\cline{2-4}
Bottom & 0, 1 & 4, 0 & 4, 0 \\
\cline{2-4}
\end{tabular}
\caption{Normal-form game for Problem 3(a) with labeled actions.}
\label{tab:3a}
\end{center}
\end{table}

For the row player, no mixed strategy over M and B can dominate T (if the column player plays L) and no mixed strategy over T and M can dominate B (if the column player plays C), but M is dominated by a mix over T and B (e.g. $p_T=\frac{2}{5}, p_B=\frac{3}{5}$).

We can now solve the subgame with row M eliminated, and see that a mix over L and C will dominate R for the column player. There are no dominant pure strategies in this subgame, so we must find the mixed strategy equilibrium. The column player mixes such that
\begin{eqnarray*}
5 p_L + 1 p_C &=& 0 p_L + 4 p_C \\
5 p_L &=& 3 p_C \\
p_L = \frac{3}{8} &,& p_C = \frac{5}{8},
\end{eqnarray*}

and the row player mixes such that

\begin{eqnarray*}
0 p_T + 1 p_B &=& 2 p_T + 0 p_B \\
p_B = \frac{2}{3} &,& p_T = \frac{1}{3},
\end{eqnarray*}
both of which make the other player indifferent. 

\framebox{\parbox{\textwidth}{The unique Nash equilibrium is the row player mixing between T and B with probability ($\frac{1}{3}, \frac{2}{3}$), and the column player mixing between L and C with probability ($\frac{3}{3}, \frac{5}{8}$).}}

\paragraph{b} 

\framebox{\parbox{0.7\textwidth}{The $2 \times 2$ game in Table \ref{tab:3b} satisfies the four properties given:}}
\begin{enumerate}
\item The row player has a weakly dominant strategy but the column player does not.
\item Top weakly dominates Bottom for the row player. Then, Left dominates Right for the column player. Top/Left is the equilibrium arrived at by iterated weak dominance.
\item Bottom/Right is a pure-strategy Nash equilibrium. The column player has no incentive to modify ($3>0$), and the row player is indifferent between top and bottom, so she also will not deviate from this equilibrium.
\item Both players prefer the second equilibrium: $2>1, 3>2$.
\end{enumerate}

\begin{table}[h!]
\begin{center}
\begin{tabular}{l|c|c|}
\multicolumn{1}{c}{} & \multicolumn{1}{c}{Left} & \multicolumn{1}{c}{Right} \\
\cline{2-3}
Top    & 1, 2 & 2, 0  \\
\cline{2-3}
Bottom & 0, 1 & 2, 3 \\
\cline{2-3}
\end{tabular}
\caption{Normal-form game for Problem 3(b) with labeled actions.}
\label{tab:3b}
\end{center}
\end{table}

To solve this, I set up the more general Table \ref{tab:3b-general}. The first constraint above gives us $w>x$ and $z>y$. The second constraint leads to $a>b$ and $c=d$. The fourth constraint gives $d>a$ and $z>w$. Combining all of these, we arrive at ordinal preferences for the row player of $c=d > a > b$ and for the column player of $z>w>y>x$. The payoffs in Table \ref{tab:3b} satisfy these rankings.

\begin{table}[h!]
\begin{center}
\begin{tabular}{l|c|c|}
\multicolumn{1}{c}{} & \multicolumn{1}{c}{Left} & \multicolumn{1}{c}{Right} \\
\cline{2-3}
Top    & $a, w$ & $c, x$  \\
\cline{2-3}
Bottom & $b, y$ & $d, z$ \\
\cline{2-3}
\end{tabular}
\caption{Normal-form game for Problem 3(b), generalized.}
\label{tab:3b-general}
\end{center}
\end{table}



\paragraph{c} \framebox{\parbox{0.9\textwidth}{The probabilities for the desired correlated equilibrium are found in Table \ref{tab:3c}.}} To find this equilibrium we must consider, for each possible action recommended to the player by the mediator:
\begin{enumerate}
\item Does the player have an incentive to obey the recommendation if the other player obeys?
\item Does the other player have an incentive to obey?
\end{enumerate}

\begin{table}[h!]
\begin{center}
\begin{tabular}{|c|c|}
\hline
0.6 & 0.2 \\
\hline
0.2 & 0  \\
\hline
\end{tabular}
\caption{Correlated equilibrium for Problem 3(c).}
\label{tab:3c}
\end{center}
\end{table}

The game is symmetric, so we can show this by only going through the steps only once for each action. Let the first action indicate the top row (left column) for the row player (column player), and the second action refer to the bottom row (right column).

Let $a$ denote the probability in the upper-left cell of the table, $b$ in the upper-right, and $c$ in the lower-left. The probability in the lower-right cell is zero as per the problem statement.

Suppose the player is signalled to use their first action. Will the player obey if her opponent does? Yes, as long as:
\begin{eqnarray*}
EU_1(\text{first action}|\text{first action signalled}) = 5\frac{a}{a+b} + 3\frac{b}{a+b} &\geq& 6\frac{a}{a+b} = EU_1(\text{second action}|\text{first action signalled}) \\
3b &\geq& a \\
a &\leq& 3b.
\end{eqnarray*}
Will the second player obey the signal? Yes, if the following constraint holds? 
\begin{eqnarray*}
EU_2(\text{first action}|\text{first action signalled}) = 5\frac{a}{a+c} + 3\frac{c}{a+c} &\geq& 6\frac{a}{a+c} = EU_2(\text{second action}|\text{first action signalled}) \\
a &\leq& 3c.
\end{eqnarray*}

What if the player is told to play the second action? If this is the case, the other player must have been told to play their first action, in which case the expected utility of playing the second action is 6, versus 5 for disobeying. 

We can exploit the symmetry of the game to solve for $a,b,$ and $c$, and the goal of maximizing $a$ subject to the constraints:
\begin{eqnarray*}
a+b+c &=& 1 \\
b &=& c \\
a &=& 3b.
\end{eqnarray*}
From this, we arrive at $a=\frac{3}{5}, b=\frac{1}{5}, c=\frac{1}{5}$, as shown in Table \ref{tab:3c}.

\end{answer}


\begin{answer}{4}
\paragraph{a} The normal-form representation of this game is presented in Table \ref{tab:4a}, where ``LR'' for the column player denotes the second player playing L if the first player played L or M, and the second player playing R if the first player played R.

\begin{table}[h!]
\begin{center}
\begin{tabular}{l|c|c|c|c|}
\multicolumn{1}{c}{} & \multicolumn{1}{c}{LL} & \multicolumn{1}{c}{LR} & \multicolumn{1}{c}{RL} & \multicolumn{1}{c}{RR}   \\
\cline{2-5}
L & 5, 2 & 5, 2 & 0, 3 & 0, 3 \\
\cline{2-5}
M & 0, 3 & 0, 3 & 5, 2 & 5, 2 \\
\cline{2-5}
R & 3, 2 & 1, 1 & 3, 2 & 1, 1 \\
\cline{2-5}
\end{tabular}
\caption{Normal-form game for Problem 4(a) with Player 1 on the row and Player 2 on the column.}
\label{tab:4a}
\end{center}
\end{table}


\paragraph{b} We wish to find a Nash equilibrium in which player 1 sometimes plays L ($P_{1,L}>0$). To do so, we must find mixing proportions such that player 2 is indifferent between L and R when in the first information set (denoted L/M):


\begin{eqnarray*}
2 P_{1,L} + 3 P_{1,M} &=& 3 P_{1,L} + 2 P+{1,M} \\
P_{1,L} &=& P_{1,M},
\end{eqnarray*}

which means that the first player must play L and M with equal probability. How should the second player mix between L and R in the L/M information set?

\begin{eqnarray*}
5 P_{2,L}|L/M + 0 P_{2,R|L/M} &=& 0 P_{2,L|L/M} + 5 P_{2,R|L/M} \\
P_{2,L|L/M} + P_{2,R|L/M} & 1 \\
P_{2,L|L/M} = P_{2,R|L/M} &=& \frac{1}{2}
\end{eqnarray*}

When in the L/M information set, the second player randomizes with equal probability between playing L and R. If the second player does this, their expected utility of playing L or M is 2.5. Thus, player 2 must also randomize between L and R if the first player played right (otherwise the first player will play right and the second player will play left, with player 1 getting a utility of $3>2.5$). Specifically, player 2 randomizes in this information set such that:

\begin{eqnarray*}
3 P_{2,L|R} + 1 P_{2,R|R} &=& 2.5 \\
P_{2,L|R} + P_{2,R|R} &=& 1 \\
2 P_{2,L|R} &=& 1.5 \\
P_{2,L|R} &=& \frac{3}{4} \\
P_{2,R|R} &=& \frac{1}{4}
\end{eqnarray*}

When player 1 has played R, player 2 randomizes between L and R with probability $(\frac{3}{4}, \frac{1}{4}$. Although this lowers player 2's expected utility in this information set, it makes player 1 indifferent between L, M, and R, which is better for player 2 overall. 

\framebox{\parbox{0.9\textwidth}{A Nash equilibrium in which player 1 sometimes plays left is \\
$(P_{1,L}=P_{1,M}=P_{1,R} = \frac{1}{3}, P_{2,L|L/M}=P_{2,R|L/M}=\frac{1}{2}, P_{2,L|R}=\frac{3}{4}, P_{2,R|R}=\frac{1}{4}$).}}

\paragraph{c} This game has two non-trivial subgames: the one in which player 1 has played R and player 2 plays L, and the one represented in Table \ref{tab:4c}. In the latter subgame, there is no pure strategy equilibrium and the mixed strategy equilibrium is $P_{1,L}=P_{1,M}=P_{2,L}=P_{2,R}=\frac{1}{2}$, as discussed above. This subgame perfect Nash equilibrium in which player 1 always plays R and player 2 plays L in the second information set and mixes equally between L and R in the first information set can be denoted \\
\framebox{\parbox{0.9\textwidth}{$(P_{1,L}=P_{1,M}=0, P_{1,R} = 1, P_{2,L|L/M}=P_{2,R|L/M}=\frac{1}{2}, P_{2,L|R}=1, P_{2,R|R}=0$).}}

\begin{table}[h!]
\begin{center}
\begin{tabular}{l|c|c|}
\multicolumn{1}{c}{} & \multicolumn{1}{c}{L} & \multicolumn{1}{c}{R} \\
\cline{2-3}
L & 5, 2 & 0, 3 \\
\cline{2-3}
M & 0, 3 & 5, 2 \\
\cline{2-3}
\end{tabular}
\caption{One subgame for Problem 4(c) with Player 1 on the row and Player 2 on the column.}
\label{tab:4c}
\end{center}
\end{table}

\end{answer}

\end{document}
