\documentclass[12pt,letterpaper]{article}
\usepackage{amsmath,amsthm,amsfonts,amssymb,amscd}
\usepackage{fullpage}
\usepackage{graphicx}
\usepackage{lastpage}
\usepackage{enumerate}
\usepackage{fancyhdr}
\usepackage{hyperref}
\usepackage{listings}
\usepackage{mathrsfs}
\usepackage{xcolor}
\usepackage[margin=3cm]{geometry}
\setlength{\parindent}{0.0in}
\setlength{\parskip}{0.05in}

% Edit these as appropriate
\newcommand\course{CS590.04}
\newcommand\semester{Spring 2014}     % <-- current semester
\newcommand\hwnum{1}                  % <-- homework number
\newcommand\yourname{Matt Dickenson} % <-- your name
\newcommand\login{mcd31}           % <-- your NetID
\newcommand\hwdate{Due: 28 February, 2014}           % <-- HW due date

\newenvironment{answer}[1]{
  \subsubsection*{Problem #1}
}


\pagestyle{fancyplain}
\headheight 35pt
\lhead{\yourname\ \texttt{\login}\\\course\ --- \semester}
\chead{\textbf{\Large Homework \hwnum}}
\rhead{\hwdate}
\headsep 10pt

\begin{document}

% \noindent \emph{Homework Notes:} 

\begin{answer}{1}
\paragraph{a} 

\framebox{\parbox{0.85\textwidth}{Borda, Copeland, and Simpson are rules that satisfy the OCO criterion.}} 

For Borda, let $B(c_i)$ denote the total Borda score for candidate $c_i$, and $B_j(c_i)$ denote the portion of the Borda score for $c_i$ resulting from the $j^{th}$ voter's ranking. With two opposite votes $v_1=(c_1, c_2, \ldots, c_m)$ and $v_2=(c_m, c_{m-1}, \ldots, c_1)$ the Borda scores are:
\begin{eqnarray*}
B(c_1) &=& B_1(c_1) + B_2(c_1) = (m-1) + 0 = m-1 \\
B(c_2) &=& B_1(c_2) + B_2(c_2) = (m-2) + 1 = m-1 \\
&\ldots& \\
B(m) &=& B_1(m) + B_2(m) = 0 + (m-1) = m-1
\end{eqnarray*}
Thus, the Borda scores for opposite votes cancel out and do not affect the winner when they are combined with another set of votes.

The Copeland and Simpson rules both depend on pairwise comparisons. For all pairs of opposite votes and each pair of candidates $c_i \neq c_j$, $c_i$ and $c_j$ will each win one pairwise comparison and lose the other. This causes the two rules to satisfy OCO. 

% it seems that Bucklin and veto would also satisfy OCO

\framebox{\parbox{1.05\textwidth}{Approval, Plurality, and Single Transferable Vote (STV) do not satisfy the OCO criterion.}} 

In an approval vote setting, the two opposite voters will not necessarily both approve all candidates preferred to their median candidate, $c_{m/2}$. If the first voter approves her preferred $k$ candidates and the second voter approves his preferred $l$ candidates ($l>k$ without loss of generality), then $l-k$ candidates ($c_{m+(k-l)/2}, \ldots, c_{m+(l-k)/2}$ will receive approval votes from both voters, which could affect the outcome of the overall election. 

When plurality voting is being used, as long as the winner without the opposite votes $c_{-1,-m}$ is not either of the opposite voters most preferred candidates ($c_1$ and $c_m$, respectively) and there are at least two non-oppositve voters, the opposite votes will not affect the election. This is because $c_{-1,-m}$ will have two or more votes and a single vote from each of the two opposite voters cannot tip the scales. By similar logic, both of the opposite voters must prefer $c_{-1,-m}$ to their opposite's most preferred candidate, so this candidate will still beat $c_1$ and $c_m$ in some elections using STV.


\paragraph{b} \framebox{\parbox{0.85\textwidth}{Borda, Copeland, and Simpson are rules that satisfy the CCO criterion,}} for similar reasons as those discussed above. 
% it seems that veto and plurality also satisfy CCO

\framebox{\parbox{0.7\textwidth}{Approval, Bucklin, and STV do not satisfy the CCO criterion.}} The reasoning for Approval is the similar to the explanation above. STV can fail to satisfy CCO because the ranks in the original election (without the addition of cycles) may be different from the cyclic votes. For example, if the cycles are over $c_1>c_2> \ldots > c_m$, some of the other voters may prefer $c_m>c_2<c_1>...$, which could affect the outcome when using STV.

To see that the Bucklin rule can fail to satisfy CCO, suppose that there are three candidates $c_1,c_2,c_3$. In the original election (without adding sets of voters with cyclical preferences), suppose $x$ voters rank the candidates $c_1,c_2,c_3$ and $y$ voters rank the candidates $c_2,c_1,c_3$. Candidate $c_1$ wins as long as $x+y>y$, i.e. if $y>1$. Suppose we add $n$ cycles of votes (i.e. $3n$ new votes). Candidate $c_2$ now wins as long as $n>x$.


\paragraph{c} \framebox{\parbox{0.9\textwidth}{Borda, Veto, Plurality, Bucklin, and Copeland all satisfy the OCCO criterion. Approval does not.}} 

With opposite cycles, all candidates will receive the same Borda score because all opposites are included. Plurality and veto allow opposite cycles to cancel because every candidate will appear in first or last (respectively) an equal number of times. Similarly for Bucklin and Copeland, all candidates in the opposite cycles multiset will win an equal number of pairwise elections.

Approval voting does not satisfy the OCCO criterion because voters can choose $m$ subsets of the $m$ candidates to approve or disapprove, which can make it arbitrarily difficult to interpret their preferences. In some pathological cases the opposite cycles cancel out (every voter approves, or every voter disapproves, of every candidate), but in many cases similar to those discussed above they do not.


\paragraph{d} \framebox{\parbox{0.95\textwidth}{OCO is stronger than OCCO (it seems that Plurality, e.g., would satisfy OCCO but not OCO). CCO is also stronger than OCCO (e.g. Bucklin seems to satisfy OCCO but not CCO). OCO and CCO are incomparable (Bucklin satisfies OCO but not CCO; Plurality satisfies CCO but not OCO).}} 

\end{answer}

\begin{answer}{2}

% \lstinputlisting[caption=MathProg Code]{../code/hw1-util-estimator.mod}

\paragraph{a} \framebox{\parbox{0.95\textwidth}{The optimal allocation is to give rackets to Alice ($U_A=8$), Bob ($U_B=8$), and Carol ($U_C=4$) for a total utility of 20. The VCG payments are 7 for Alice and Bob, 3 for Carol, and zero for Daniel, Eva, and Frank.}}

% \paragraph{b} In general (general graphs, bids, numbers of rackets), is the problem of finding the optimal allocation solvable in polynomial time, or NP-hard? (Hint: think about the Clique problem (which is almost the same as the Independent Set problem).)

\paragraph{b} \framebox{\parbox{0.95\textwidth}{The problem of optimal racket allocation (with general graphs, bids, and numbers of rackets) is NP-hard.}}

% explain how independent set reduces to clique

For general graphs $G$ with $n$ nodes, there are $n \choose k$ possible allocations of $k$ rackets. Because of the way that the problem is defined, the optimal allocation will always allocate all of the rackets, assuming $k<n$ (i.e. we can do no worse by allocating all rackets, even if the receiver has no neighbors with whom she can play). Furthermore, we are better off allocating rackets to nodes whose neighbors also have rackets, so that they can play together. The solution to the racket problem, then, will always be a maximally connected subgraph of $G$. 

If we were given a graph $G=\{V,E\}$ and asked to perform Clique, we could instantiate the Racket problem with the graph $G$, arbitrary valuations (e.g. $1$), and $k$ rackets. (This would be the $k$-clique problem; more generally we could find the maximal clique and reject the solution if its size were greater than $k$). Because Clique is NP-complete (Karp 1972) and can be reduced to the Racket problem, the Racket problem is also NP-complete. 

\paragraph{c} \framebox{\parbox{0.95\textwidth}{The optimal allocation is to give rackets to Alice ($U_A=4$), Bob ($U_B=1$), Carol ($U_C=5$), and Daniel ($U_D=5$) for a total utility of 15. The VCG payments are 3 for Alice, 2 for Carol, and zero for everyone else.}}

\paragraph{d} \framebox{\parbox{0.95\textwidth}{For general graphs and bids, the problem of finding the optimal outcome is solvable in polynomial time.}} Because each racket recipient can play with at most one neighbor, we can reformulate the problem so that the sum of a pair of valuations is the weight on the edge between two nodes (e.g. the weight on the Alice-Bob edge is 5 and the weight for Alice-Carol is 7 in the example given). In this way we can reduce racket allocation with a fixed number of rackets to weighted bipartite matching, which is solvable in $O(V^2 E)$ time for a graph with $V$ nodes (vertices) and $E$ edges. 

% (13 points) In general (general graphs and bids), is the problem of finding the optimal outcome solvable in polynomial time, or NP-hard? (Hint: think about the Maximum-Weighted-Matching problem. Keep in mind that the number of rackets is limited, though.)

% http://en.wikipedia.org/wiki/Matching_(graph_theory)#In_general_graphs
% http://en.wikipedia.org/wiki/Matching_(graph_theory)#In_weighted_bipartite_graphs
% http://en.wikipedia.org/wiki/Hungarian_algorithm
\end{answer}


\end{document}
